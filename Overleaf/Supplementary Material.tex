%%%%%%%%%%%%%%%%%%%%%%%%%%%%%%%%%%%%%%%%%%%%%%%%%%%%%%%
%                   File: OSAmeetings.tex             %
%                  Date: 20 September 2021            %
%                                                     %
%     For preparing LaTeX manuscripts for submission  %
%       submission to Optica meetings and conferences %
%                                                     %
%       (c) 2021 Optica                               %
%%%%%%%%%%%%%%%%%%%%%%%%%%%%%%%%%%%%%%%%%%%%%%%%%%%%%%%

\documentclass[letterpaper,10pt]{article} 
%% if A4 paper needed, change letterpaper to A4

\usepackage{osameet3} %% use version 3 for proper copyright statement

%% standard packages and arguments should be modified as needed
% Useful packages
\usepackage{mathrsfs}
\usepackage{relsize}
% \newcolumntype{C}{>{\centering\arraybackslash}X}
\usepackage[labelformat = simple, position=top]{subcaption} %% provides subfigure
\renewcommand{\figurename}{Figure}
\renewcommand{\thefigure}{S\arabic{figure}}
\renewcommand\thesubfigure{(\alph{subfigure})} %% add parentheses around subfigure references
\usepackage{caption}
\captionsetup{skip=5pt}
\usepackage[table,xcdraw,dvipsnames]{xcolor}
\definecolor{dimgray}{rgb}{0.41, 0.41, 0.41}
\usepackage{amsmath,amssymb}
\usepackage{float} %make the table after the text
\usepackage{placeins} %make the table after the text
\usepackage{color}
\usepackage{amssymb}
\usepackage{longtable}
\usepackage{tabulary}
\usepackage{makecell}
\usepackage{graphicx}
\usepackage{comment}
\usepackage{url}
\usepackage{booktabs}
\usepackage{enumerate}
\usepackage{multirow,array}
\usepackage{amsthm}
\numberwithin{equation}{section}
\usepackage{amsmath}
\usepackage{xr-hyper}
\usepackage{appendix}
\usepackage[colorlinks=true,bookmarks=false,citecolor=blue,urlcolor=blue, allcolors=blue]{hyperref}
\usepackage{nccmath} % fomula left align
\usepackage{lipsum}

\usepackage{xr}
\makeatletter
\newcommand*{\addFileDependency}[1]{% argument=file name and extension
  \typeout{(#1)}% latexmk will find this if $recorder=0 (however, in that case, it will ignore #1 if it is a .aux or .pdf file etc and it exists! if it doesn't exist, it will appear in the list of dependents regardless)
  \@addtofilelist{#1}% if you want it to appear in \listfiles, not really necessary and latexmk doesn't use this
  \IfFileExists{#1}{}{\typeout{No file #1.}}% latexmk will find this message if #1 doesn't exist (yet)
}
\makeatother

\newcommand*{\myexternaldocument}[1]{%
    \externaldocument{#1}%
    \addFileDependency{#1.tex}%
    \addFileDependency{#1.aux}%
}
%%% END HELPER CODE

% put all the external documents here!
\myexternaldocument{./elsarticle-template-num}

\renewcommand{\theequation}{S.\arabic{equation}}

\newtheorem{theorem}{Theorem}[section]
\newtheorem{corollary}{Corollary}[theorem]
\newtheorem{lemma}[theorem]{Lemma}

\begin{document}

\title{Supplementary Material of\\``The dynamics of corruption under an optional external supervision service"}

\author{Xin Zhou*, Adam Belloum, Michael H. Lees, Tom van Engers, Cees de Laat}

\address{University of Amsterdam, Amsterdam, 1098XH, Netherlands}
%Uncomment the following line to override copyright year from the default current year.
\copyrightyear{2022}
\renewcommand{\thesection}{\Alph{section}}

\section{Supplement of Analytical Results}\label{AppendixA} 
\subsection{The stability of fixed points}\label{AppendixA1} % *
% $ \mathbf{x_{1}}^{*} $ *
    Based on the replicator-equations(\ref{R-E}), it is easy to write $ \dot{x} $ in the formation:
    \begin{equation}
        \label{R-E-Expanding}
        \begin{split}
            \left[ \begin{array}{c}
                \dot{x_{1}}\\
                \dot{x_{2}}\\
                \dot{x_{3}}\\
            \end{array}\right] = 
            \left[ \begin{array}{c}
                x_{1}((A_{c}\mathbf{x})_{1}-\mathbf{x}^{\top}A_{c}\mathbf{x})\\
                x_{2}((A_{c}\mathbf{x})_{2}-\mathbf{x}^{\top}A_{c}\mathbf{x})\\
                x_{3}((A_{c}\mathbf{x})_{3}-\mathbf{x}^{\top}A_{c}\mathbf{x})\\
            \end{array}\right],
        \end{split}
    \end{equation}
    the Jacob matrix of (\ref{R-E-Expanding}) is,
    \begin{equation}
        \label{Jacob-M}
        \begin{split}
            \mathbf{J} = \left[ \begin{array}{c c c}
                \frac{tial \dot{x_{1}}}{tial x_{1}} & \frac{tial \dot{x_{1}}}{tial x_{2}} & \frac{tial \dot{x_{1}}}{tial x_{3}} \\[0.5em]
                \frac{tial \dot{x_{2}}}{tial x_{1}} & \frac{tial \dot{x_{2}}}{tial x_{2}} & \frac{tial \dot{x_{2}}}{tial x_{3}} \\[0.5em]
                \frac{tial \dot{x_{3}}}{tial x_{1}} & \frac{tial \dot{x_{3}}}{tial x_{2}} & \frac{tial \dot{x_{3}}}{tial x_{3}} \\[0.5em]
            \end{array}\right].
        \end{split}
    \end{equation}
The real part of the eigenvalues of $ \mathbf{J} $ at the fixed point decides the stability of the fixed point. If the eigenvalues have negative real parts, then the fixed point is asymptotically stable \cite{sandholm2010local, hines1980three}.

Under heavy punishment, $ f = 2 $, facing pure corrupt rule enforcers, when $ a = 0.1, b=c=0.5, C_{0} = B = 0.2 $, the internal fixed point $ \mathbf{x_1}^* = (3/20, 201/260, 1/13) $. The eigenvalue of $ \mathbf{J} $ at $ \mathbf{x_1}^* $ is $ \bigl(-29/130,(42-6 \mathrm{i} \sqrt{69631})/10400,(42+6 \mathrm{i} \sqrt{69631})/10400\bigr) $. Since the real part of the second and third eigenvalue are positive, $ \mathbf{x_1}^* $ is not asymptotically stable.

Keeping the value of all the rest parameters but set $ a $ to $ 0 $, $ \mathbf{x_1}^* = (3/20, 17/20, 0) $, and the eigenvalue of $ \mathbf{J} $ at $ \mathbf{x_1}^* $ is $ (-3/10,0,0) $. Since all the real part of the eigenvalues are less than or equal to zero, and with no imaginary parts, $ \mathbf{x_1}^* $ is stable. Actually, as Figure~\ref{FigAppenA1} shows, every point on the edge $ C_{a}C_{\bar{a}} $ (represented by a dashed line) is stable.
    
    \begin{figure}[H]
        \centering
        \includegraphics[height = 4.2cm]{AppenA(1).jpg}
        \caption{When $ y_{1} = 2, b=c=0.5, C_{0} = B = 0.2, f = 2, a = 0 $, the original interior fixed point $ \mathbf{x_1}^* $ locates on the edge $ C_{a} C_{\bar{a}} $. All points on the dashed edge $ C_{a}C_{\bar{a}} $ are stable.}
        \label{FigAppenA1}
    \end{figure}

\subsection{Player-enforcer dynamics in an infinite and well-mixed population}\label{AppendixA2} % *
\subsubsection{The equilibrium without exploration}\label{AppendixA21}

\begin{figure}[H]
    \centering
    \begin{tabular}{c}
        \begin{subfigure}[b]{0.5\textwidth}
            \includegraphics[height=4.2cm]{AppenA(2).jpg}
            \caption{$ \mathbf{x}^{(0)} = (0.25,0.25,0.5) $, $ \mathbf{y}^{(0)} = (0.2,0.8) $, $ a = 0.1 $}
            \label{AppenA2}
        \end{subfigure}
        \begin{subfigure}[b]{0.5\textwidth}
            \includegraphics[height=4.2cm]{AppenA(3).jpg}
            \caption{$ \mathbf{x}^{(0)} = (0.25,0.25,0.5) $, $ \mathbf{y}^{(0)} = (0.2,0.8) $, $ a = 0.5 $}
            \label{AppenA3}
        \end{subfigure}
    \end{tabular}
    \caption{Player-enforcer dynamics when $ y_1^{(0)} > (B-c)/(B-f) $. Both $ \mathbf{y}^* $ and $ \mathbf{x}^*$ are reachable, $ y_{1}^* < y_{1}^{(0)} $, $ \mathbf{x}^* = (0, 1, 0) $. Compare Figure~\ref{AppenA2} and Figure~\ref{AppenA3}, it can be observed that, when $ a $ is higher, the stronger inhibition effect of $ a $ on the cautious cooperators reduces the summit of the fraction of $ C_{a} $, and accelerates the elimination of $ C_{a} $ by $ C_{\bar{a}} $. Meanwhile, the heavier punishment effect of $ a $ on $ U_{c} $ leads to a slight increase of the fraction of honest enforcers at the beginning.}
    \label{AppendBFig1}
\end{figure}

The equilibrium of the population profile of rule enforcers and players depend on the initial state of the enforcers $ \mathbf{y}^{(0)} $. When $ y_{1}^{(0)} > (B-c)/(B-f) $, $ y_{1}^* $ decreases slightly from $ y_{1}^{(0)} $ in the equilibrium, and the corresponding equilibrium of players is $ \mathbf{x}^* = (0,1,0) $. Furthermore, comparing Figure~\ref{AppenA2} and Figure~\ref{AppenA3}, it can be noticed that the value of $ a $ do not change $ \mathbf{x}^* $ and $ \mathbf{y}^* $.

However, $ a $ changes the trajectory of $ \mathbf{x} $ and $ \mathbf{y} $. Due to the inhibition effect of $ a $ on $ C_{a} $, a higher $ a $ is accompanied by a lower summit of and a steeper slop of the fraction of $ C_{a} $. We also note that in Figure~\ref{AppenA3}, the fraction of corrupt enforcers when $ a=0.5 $ is not monotonic increasing as in~\subref{AppenA2}, but first decreasing and then increasing. This result attributes to the punishment effect of $ a $ on $ U_{c} $: the higher the $ a $ is, the less $ \pi(U_{c}) $ is in the event $ (C_{a}, D)|(U_{c}) $.

    
\subsubsection{The equilibrium with exploration} \label{AppendixA22}
% *
When the players and enforcers are allowed to explore strategies randomly ($ \mu \neq 0 $ and $ v \neq 0 $), $ \mathbf{x}^* $ and $ \mathbf{y}^* $ always exist. In Figure~\ref{Fig2}, we show the player-enforcer dynamics under $ \mathbf{x}^{(0)} = (0.25, 0.25, 0.5) $, $ \mathbf{y}^{(0)} = (0.1, 0.9) $. For elaborating on the independence of the mixed strategy equilibrium to the initial state, we set a different initial state, $ \mathbf{x}^{(0)} = (1/3, 1/3, 1/3), \mathbf{y}^{(0)} = (0.9, 0.1) $, the results are shown in Figure~\ref{AppendBFig2}. Compare Figure~\ref{AppenA4} to Figure~\ref{Fig2c}, it is easy to tell that $ \mathbf{x}^* $ and $ \mathbf{y}^* $ are the same with different initial state. Analogously, the equilibrium in Figure~\ref{AppenA5} and Figure~\ref{Fig2e} are the same.

The initial state does not change the equilibrium, but influences the required time of reaching the stable state. When the distance between the initial state and the equilibrium is greater, the required time is longer. For example, the required time in Figure~\ref{AppenA5} is much longer than in Figure~\ref{Fig2e}, as $ \mathbf{y}^{(0)} = (0.9, 0.1) $ is further from $ \mathbf{y}^*\approx (0, 1) $ compared with $ \mathbf{y}^{(0)} = (0.1, 0.9) $. In contrast, $ \mathbf{y}^{0} $ in Figure~\ref{AppenA4} and Figure~\ref{Fig2c} are of an equal distance to the equilibrium $ \mathbf{y}^* \approx (0.5, 0.5) $, the required time in these two scenarios are hence similar.

\begin{figure}[H]
    \centering
    \begin{tabular}{c}
        \begin{subfigure}[b]{0.5\textwidth}
            \includegraphics[height=4.2cm]{AppenA(4).jpg}
            \caption{$ \mu = 0.01, v = 0.05 $, $ a = 0.1 $}
            \label{AppenA4}
        \end{subfigure}
        \begin{subfigure}[b]{0.5\textwidth}
            \includegraphics[height=4.2cm]{AppenA(5).jpg}
            \caption{$ \mu = 0.005, v = 0.001 $, $ a = 0.5 $}
            \label{AppenA5}
        \end{subfigure}
    \end{tabular}
    \caption{Player-enforcer dynamics with the initial state $ \mathbf{x}^{(0)} = (1/3, 1/3, 1/3), \mathbf{y}^{(0)} = (0.9,0.1) $. If players and enforcers are allowed to explore strategies randomly, both $ \mathbf{y}^* $ and $ \mathbf{x}^*$ are reachable, and they are robust to the initial state.}
    \label{AppendBFig2}
\end{figure}

\section{Supplement of Simulation Experiments Algorithms}\label{AppendixB}
\subsection{The stochastic replicator dynamics algorithm in a finite population}\label{AppendixB1}
Assume a transfer matrix $ \mathbf{M} $,

\begin{equation}
\mathbf{M} = 
    \begin{bmatrix}
    P_{1,1} & P_{1,2} & P_{1,3} \\
    P_{2,1} & P_{2,2} & P_{2,3} \\
    P_{3,1} & P_{3,2} & P_{3,3} \\
    \end{bmatrix},
    \label{transmatrix}
\end{equation}
$ P_{i,j} = \bigl[1+\exp \bigl(\mathbf{S}(\pi(S_{i})-\pi(S_{j}))\bigr)\bigr]^{-1} $, which denotes the probability that a player transform from strategy $ S_{i} $ to $ S_{j} $. The exploration matrix $ \mathbf{M_{E}} $ is:

\begin{equation}
\mathbf{M_{E}} = 
    \begin{bmatrix}
    -\mu & \mu/2 & \mu/2 \\
    \mu/2 & -\mu & \mu/2 \\
    \mu/2 & \mu/2 & -\mu \\
    \end{bmatrix}.
\end{equation}
Let $ \lVert \mathbf{M} \rVert $ be the row normalized matrix of $ \mathbf{M} $, in $ \lVert \mathbf{M} \rVert $, $ \sum_{j = 1}^{j = 3} P_{i,j} = 1 $. The population profile at the next time step is:

\begin{equation}
    \mathbf{x}^{(t+1)} = \mathbf{x}^{(t)} ( \lVert \mathbf{M} \rVert+\mathbf{M_{E}}). \label{update_x}
\end{equation}
Thus, if $ (\lVert \mathbf{M} \rVert+\mathbf{M_{E}}) $ is fixed, meanwhile $ \mathbf{x}^{(t+1)} = \mathbf{x}^{(t)} $, then $ \mathbf{x}^{(t)} $ reaches $ \mathbf{x}^* $.

\begin{theorem}\label{Theo: leftEigen}
If the stable state exists, $ \mathbf{x}^* $ and $ \mathbf{y}^* $ must be the left eigenvectors of the corresponding constant matrix $ \lVert \mathbf{M} \rVert+\mathbf{M_{E}} $.
\end{theorem}

\noindent The exploration matrix for enforcers is analogous, $ \mathbf{M_{E}} $ is adapted to:

\begin{equation}
\mathbf{M_{E}} = 
    \begin{bmatrix}
    -v & v \\
    v & -v \\
    \end{bmatrix}
\end{equation}.

\subsection{Algorithm for stochastic dynamics in a finite population}\label{AppendixB2}
\begin{longtable}{p{0.1\columnwidth} p{0.85\columnwidth}}
    \caption{The algorithm of stochastic dynamics in a finite population}
	\endfirsthead
    \endhead
		\hline
		\textbf{Input:} & $ N, b, c, C_{0}, f, B, \mathbf{S} $, the cost of external supervision service $ a $, the mutation rate of players $ \mu $, the mutation rate of rule enforcers $ v $, the initial population profile $ \mathbf{x}^{(0)} $, $ \mathbf{y}^{(0)} $, and the termination time step $ \mathbf{T} $.\\
		\textbf{Output:} & $ \mathbf{x} $, $ \mathbf{y} $.\\
		\textbf{Step 1:} & \makecell[tl]{If $ t < \mathbf{T} $, compute \#$  C_{a} $, \#$  C_{\bar{a}}$, and \#$  D $ based on $ \mathbf{x}^{(t)} $; compute \#$  U_h $ and \\ \#$  U_c $ based on $ \mathbf{y}^{(t)} $.\\
		Then match players randomly in pairs, and assign rule enforcers to those pairs.\\
		Generate the pairwise table, each row of which is composed of two players\\
		and one random assigned rule enforcer. eg. $ (C_{\bar{a}}, C_{\bar{a}}, U_{h}) $.\\
		Else go to \textbf{step 6}.}\\
		\textbf{Step 2:} & Based on the pairwise table, calculate the payoff of each player and rule enforcer. Hence, generate the \textbf{payoff table}, each row is composed of the payoff of the two players and the enforcer, eg. $ (0.3, 0.3, 0.4) $.\\
		\textbf{Step 3:} & Based on the payoff table, calculate the average payoff of strategies: $ \pi (C_{\bar{a}}) $,
		$ \pi (C_{a}) $, and $ \pi (D) $; $ \pi (U_{h}) $ and $ \pi (U_{c})$.\\
		\textbf{Step 4:} & Calculate $ \mathbf{x}^{(t+1)} $ and $ \mathbf{y}^{(t+1)} $ according to the replicator dynamics algorithm in Supplementary Material~\ref{AppendixB1}. Update $ t $ to $ t+1 $\\
		\textbf{Step 5:} & Go to step 1.\\
		\textbf{Step 6:} & End.\\
		\bottomrule
\end{longtable}


\section{Supplement of Simulation Experiments Results}\label{AppendixC}
\subsection{Proof of y*=(0.065, 0,935) with v = 0.05} \label{AppendixC1}
\begin{proof}
Assume that at time step $ t $, $ \mathbf{y} = (0.065, 0.935) $. 
With $ N = 10 $, $ M = \#U_c + \#U_h = 5 $. Among the 5 enforcers, $ \#U_h = 5*0.065 = 0 $, and $ \#U_c = 5 $. Naturally, $ \pi(U_{h}) = 0 $, and $ \pi(U_{c}) = 2C_{0}(1-x_{1}x_{3})+2Bx_{3}-2ax_{1}x_{3} = 0.2214 \geq 0 $. According to the stochastic dynamics algorithm in Supplementary Material~\ref{AppendixB1},

\begin{equation}
    \mathbf{M} = 
        \begin{bmatrix}
        1/2 & 1 \\
        0 & 1/2 \\
    \end{bmatrix}.
\end{equation}
Since $ \pi(U_h) \equiv 0 $ and $ \pi(U_h) $ is always greater than $ 0 $, $ \mathbf{M} $ is constant. Then we have,
\begin{equation}
    \lVert \mathbf{M} \rVert = 
        \begin{bmatrix}
        1/3 & 2/3 \\
        0 & 1
        \end{bmatrix},
\end{equation}
with
\begin{equation}
    \mathbf{M_{E}} = 
        \begin{bmatrix}
        -0.05 & 0.05 \\
        0.05 & -0.05
        \end{bmatrix},
\end{equation}
Then we have
\begin{equation}
    (\lVert \mathbf{M} \rVert+\mathbf{M_{E}}) = 
        \begin{bmatrix}
        0.2833 & 0.7167 \\
        0.05  & 0.95
        \end{bmatrix},
\end{equation}
whose left eigenvector is $ (0.065, 0,935) $, according to Theorem~\ref{Theo: leftEigen}, $ \mathbf{y}^* = (0.065, 0,935) $ is a stable state.
\end{proof}

\subsection{The sustainability of players' strategy profile in a finite market}\label{AppendixC2}
\subsubsection{x*=(0, 1, 0) when the fraction of honest enforcers no less than 0.5} \label{AppendixC21}

\begin{lemma}
    In a small or medium scale market, $ \mathbf{x}^* = (0,1,0) $ if $ y_{1}^* \geq 0.5 $.
\end{lemma}

In Section~\ref{AnaResultNoExploration}, we have shown that $ \mathbf{x}^* = (0, 1, 0) $ if $ y_{1}^{*} > 2(B - c)/(2B - 3f) $ in an infinite and well-mixed population. When allowing for random exploration, $ \mathbf{x}^* $ cannot reach $ (0, 1, 0) $, but $ C_{\bar{a}} $ becomes the majority and $ x_{2}^* \approx 1 $. If $ N $ is finite, the transfer matrix (\ref{transmatrix}) is uncertain because of the chance event can lead to different $ \pi(S_{i}) $. Thus, it is hard to prove $ \mathbf{x}^* = (0,1,0) $ by calculating the payoff matrix. Nevertheless, we can prove by reasoning the required condition of reaching $ \mathbf{x}^* $. Note that the precondition of reaching an equilibrium is that the mutation rate is low enough such that once the equilibrium has been reached, other strategies cannot invade. Hence, we only consider the low exploration group.

\begin{proof}
    It is known that homogeneous $ D $ or $ C_{a} $ is unstable ($ a>0 $), the only possible equilibrium for players is homogeneous $ C_{\bar{a}} $. However, the more trusting cooperators existing in the market, the more likely defectors invade, since defectors gains more in the scenario $ (C_{\bar{a}}, D) $ than any other scenarios. Therefore, in a finite market that is composed of one defector and $ N-1 $ trusting cooperators, the expected value of $ \pi(D) $ get maximized.
    
    $ E(\pi(D)) = y_{1}(b+c-C_0-f) + y_{2}(b+c-C_0-B) $, and similarly $ E(\pi(C_{\bar{a}})) = y_{1}(-c+f/2-C_0) + y_{2}(-c-C_0) $. When $ y_{1}^* \geq 0.5 $. We have $ E(\pi(D)) < E(\pi(C_{\bar{a}})) $, therefore, $ x_{3} $ tends to decrease and $ x_{2} $ tends to increase. As long as $ x_{2} \geq 0.95 $, the market is composed of homogeneous $ C_{\bar{a}} $, meanwhile the small mutation rate cannot bring in defectors. $ \mathbf{x}^* = (0,1,0) $ is then reached.
\end{proof}

\subsubsection{Dynamic patterns of x*=(0, 1, 0) when y* = (0,1)}\label{AppendixC22}
  
\textbf{(1) $ N = 10, \mu = 0.01 $}\label{AppendixC221}

\begin{lemma}
    In a small scale market, $ \mathbf{x}^* = (0,1,0) $ when $ \mu = 0.01 $.
\end{lemma}

Theoretically, surrounded by corrupt enforcers, $ \mathbf{x}^* $ is unreachable, but when the market is small, $ \mathbf{x}^* = (0,1,0) $ can happen. When $ N = 10 $ and $ \mu = 0.01 $, $ \mathbf{y}^* \in \{(0, 1), (0.5, 0.5), (1, 0)\} $. We have proved that when $ y_{1}^* \geq 0.5 $, $ \mathbf{x}^* = (0, 1, 0) $. Now we prove that even with $ y_{1}^* = (0, 1) $, $ \mathbf{x}^* $ is still reachable.

\begin{proof}
    Let us assume all rule enforcers are corrupt ($ y_{1}^* = 0 $). Still, the fraction of $ D $ only increases when the event $ (C_{\bar{a}}, D)|(U_{c}) $ happens. Otherwise, the fraction of defectors tends to decrease, and during which the fraction of cooperators increases. Among the cooperators, the trusting cooperators eliminate the cautious ones, until take over the market ($ \mathbf{x}^* = (0,1,0) $).
    
    Despite the chance of event $ (C_{\bar{a}}, D) $ not happening continuously is low, especially when trusting cooperators are the majority, such low chance event would eventually happen as long as time is long enough. Additionally, the small exploration rate ($ \mu = 0.01 $) leaves no chance to the defectors to invade once the equilibrium has been reached. Therefore, $ \mathbf{x}^* = (0, 1, 0) $ in small scale markets when $ \mu = 0.01 $ .
\end{proof}

\noindent
\textbf{(2) $ N = 100, \mu = 0.001 $}\label{AppendixC222}

Usually, in a medium scale market, $ \mathbf{x}^* $ does not exist when $ \mathbf{y}^* = (0, 1) $. Table~\ref{Table: X|maxX2} shows the fraction of cautious cooperators ($ x_{1} $) and of the defectors ($ x_{3} $) when the fraction of trusting cooperators ($ x_{2} $) reaches its summit facing pure corrupt enforcers. The results are the average of the 500 times repeated experiments.

When $ a < 0.2 $, $ x_{1} + x_{3} < 0.005 $, then the population has evolved into $ \mathbf{x}^* = (0, 1, 0) $ where \#$ C_{a} = $ \# $ D = 0 $. However, when $ a \geq 0.3 $, the minimum fraction of defectors is $ 0.005 $. It means that there is at least one defector remains in the market, once the event $ (C_{\bar{a}}, D) $ happens, $ x_{2} $ drops from its summit. Accordingly, $ \mathbf{x}^*= (0,1,0) $ cannot be reached.

\begin{table}[H]
\centering
    \caption{$ (x_{1},x_{3}) | \bigl(x_{2} = \max(x_{2}), \mathbf{y}^* = (0,1)\bigr) $}
    \label{Table: X|maxX2}
    \begin{tabular}{l|cccc}
        \hline
                & \multicolumn{2}{c}{$ \mu = 0.001, v = 0.005 $} & \multicolumn{2}{c}{$ \mu = 0.005, v = 0.001 $} \\ \cline{2-5} 
                & $ x_{1} $ & $ x_{3} $ & $ x_{1} $ & $ x_{3} $ \\ \hline
        a = 0.1 & 0.001     & 0.001     & 0.006     & 0.003    \\
        a = 0.2 & 0.001     & 0.001     & 0.006     & 0.003    \\
        a = 0.3 & 0.006     & 0.005     & 0.008     & 0.007    \\
        a = 0.4 & 0.005     & 0.009     & 0.008     & 0.008    \\
        a = 0.5 & 0.006     & 0.009     & 0.007     & 0.009    \\
        \hline
    \end{tabular}
\end{table}

However, there are 28 cases among the 500 times repeated experiments with $ a = 0.3 $ unexpectedly reaches $ \mathbf{x}^*=(0,1,0) $. The 500 repeated experiments are marked from 1 to 500, the numbers of the 28 cases are: 21, 23, 125, 127, 147, 153, 162, 163, 167, 243, 250, 260, 272, 296, 302, 324, 331, 350, 378, 384, 393, 396, 397, 404, 440, 443, 458, 469. In table~\ref{Table: x272}, we present the stochastic dynamics of $ \mathbf{x} $ during time step 1146 to 1151, for elaborating on these uncommon cases.

\begin{table}[h]
    \centering
    \caption{Stochastic dynamics of $ \mathbf{x} $ in the $ 272^{nd} $ experiment}
    \label{Table: x272}
    \begin{tabular}{c|cccccccc}
    \hline
        Time Step & 1146   & 1147   & 1148   & 1149   & 1150   & 1151   & 1152   & 1153   \\ \hline
        $ x_{1} $ & 0.2828 & 0.1001 & 0.0262 & 0.0198 & 0.0093 & 0.0032 & 0.0021 & 0.0016 \\
        $ x_{2} $ & 0.7027 & 0.8965 & 0.9474 & 0.9745 & 0.9890 & 0.9935 & 0.9958 & 0.9969 \\
        $ x_{3} $ & 0.0145 & 0.0034 & 0.0264 & 0.0057 & 0.0016 & 0.0032 & 0.0021 & 0.0016 \\ \hline
    \end{tabular}
\end{table}

At step 1147, we notice that the fraction of trusting cooperators ($ x_{2} $) increases to 89.65\%, the rest of the players are cautious cooperators. In the next time step 1148, $ x_{2} $ further increases to 97.45\% after eliminating the cautious cooperators($ C_{a} $), but the exploration rate brings a few defectors to the market. From $ \mathbf{x}^{(1149)} $, it can be inferred that all the defectors are paired with $ C_{a} $, which drives $ x_{2} $ further increases to 97.45\%. The market then is composed of 97 $ C_{\bar{a}} $, 2 $ C_{a} $, and 1 $ D $; the only defector is paired with one of the cautious cooperators, namely, the event $ (C_{\bar{a}}, D) $ never happens. $ x_{2} $ hence increases further to 98.9\% at time step 1150. At this time step, the market has 99 $ C_{\bar{a}} $ and one $ C_{a} $. Since $ C_{\bar{a}} $ is the dominant strategy, $ x_{2} $ increases again at step 1151 to 99.35\%, where there are 99 $ C_{\bar{a}} $. The remaining one player theoretically can be either $ C_{a} $ or $ D $ with an equal chance. In this experiment, it happens to be $ C_{a} $, then $ x_{2} $ increases and finally reaches $ 99.58\% $. Henceforth, all the 100 participants are trusting cooperators. 

In the whole process, \# $ D $ is less than 3, and \#$ C_{\bar{a}} $ increases from 90 to 100, none of the defectors is paired with $ C_{a} $. However, the probability of the event $ (D, C_{a}) $ not happening continuously is quite low, especially when $ x_{2} $ is high and keeps increasing. Therefore, $ \mathbf{x}^* = (0, 1, 0) $ rarely happens when $ \mathbf{y}^* = (0, 1) $ in medium scale markets.

\subsection{Analysis of the cycle length} \label{AppendixC3}
\subsubsection{Cycle length within large scale markets}\label{AppendixC3_1000}
% The influence of mutation rate
Figure~\ref{AppenC31} gives the average cycle length of strategy $ C_{a} $ and $ U_{c} $. The numerical results provide hints to understand this phenomenon. From section~\ref{AnaResultWithExploration}, we know $ x_{2}^* $ is larger when $ v > \mu $ than when $ v < \mu $ (Figure~\ref{Fig2}). Namely, when $ v > \mu $, $ C_{\bar{a}} $ always tends to dominant the market, which makes $ D $ more likely to invade, thus the stable oscillation of the system is accelerated; but when $ v < \mu $, there are more $ D $ and $ C_{a} $ in the market which improves the chance of events $ (D, D) $ and $ (C_{a}, D) $, and then delays the growth of $ D $ and $ U_{c} $. That's why when $ v > \mu $, $ \mathscr{C}(S_{i}) $ is larger. Holding the relative value of $ \mu $ and $ v $, the larger the absolute value of $ \mu $ is, the larger $ \mathscr{C}(S_{i}) $ will be. The reason for this is that a high mutation rate can cause more irrational agents to choose the less dominated strategies, which decelerates the elimination process.

\begin{figure}[H]
    \begin{center}
        \includegraphics[height=3.1in]{AppenC(1).jpg}
    \end{center}
\caption{The cycle length of strategies $ C_{a} $ and $ D $ within large scale markets where $ N = 1000 $. Both the mutation rate and the cost of external supervision services can influence $ \mathscr{C}(C_{a}) $ and $ \mathscr{C}(U_{c}) $. When $ v > \mu $, $ \mathscr{C}(S_{i}) $ is larger, since the potential $ x_{1}^* $ is one under $ v > \mu $, defectors are more likely to invade, which drives the evolution to move into the next period. The absolute value of exploration rates also changes the cycle length. A high mutation rate decelerates the dynamic process by introducing more irrational agents, which prolongs the cycle length. In terms of $ a $, it changes $ \mathscr{C}(S_{i}) $ indirectly through its inhibition effect on $ C_{a} $ and punishment effect on $ U_{c} $. Both effects can shorten or prolong the period, the net result of the two effects depends on the value of $ a $. This complicated mechanism leads to a non-monotonic relationship between $ \mathscr{C}(S_{i}) $ and $ a $.} 
\label{AppenC31}
\end{figure}

% The inlfuence of a on Ca
As to the cost of external supervision services, it has an inhibition effect on $ C_{a} $ and a punishment effect on $ D $, these two effects indirectly change $ \mathscr{C}(C_{a}) $ and $ \mathscr{C}(D) $. For $ C_{a} $, the inhibition effect on the one hand shortens $ \mathscr{C}(C_{a}) $ by 1) terminating the growth of $ C_{a} $ in advance and limiting its summit, or rather that, once $ C_{a} $ eliminates $ D $ to a certain level, $ x_{1} $ stops growing and immediately drops (Compare Figure~\ref{Fig1e} to Figure~\ref{Fig1d}); 2) strengthening the dominance of $ C_{\bar{a}} $ to $ C_{a} $, $ C_{\bar{a}} $ then eliminates $ C_{a} $ earlier and faster. On the other hand, the inhibition effect prolongs $ \mathscr{C}(C_{a}) $ by lifting the required fraction of defectors for $ C_{a} $ to start growing, which delays the growth of $ x_{1} $. These two conflict consequences make $ \mathscr{C}(C_{a}) $ change non-monotonically as $ a $ increases.

When $ a $ increases from 0.1 to 0.2, the former consequence is stronger, and $ \mathscr{C}(C_{a}) $ decreases. This result is also confirmed in Figure~\ref{Fig7}. In Figure~\ref{Fig7a}, the significant decrease of $ Var(C_{a}) $ when $ a $ increases to 0.2 indicates the decrease of the summit of $ C_{a} $. The increase of $ cov(C_{a}, C_{\bar{a}}) $ from -0.0075 to -0.065 in Figure~\ref{Fig7b} evidences the shorter horizontal phase shift between $ x_{1} $ and $ x_{2} $, which is caused by the earlier and faster elimination of $ C_{a} $ by $ C_{\bar{a}} $. However, when $ a $ further increases, the latter consequence becomes stronger, and $ \mathscr{C}(C_{a}) $ becomes longer (Compare Figure~\ref{Fig1e} to~\subref{Fig1d}). The decrease of $ cov(C_{\bar{a}}, D) $ when $ a $ increases from 0.2 to 0.5 is exactly the consequence of $ x_{1} $ longer staying at a low level, because when the market contains very few $ C_{a} $, the event $ (C_{\bar{a}}, D) $ is more likely to happen, which strengthens the negative correlation between $ C_{\bar{a}} $ and $ D $.

% Special points brought by mu = 0.05
The trend of $ \mathscr{C}(C_{a}) $ under $ \mu = 0.05, v = 0.01 $ is different from under other mutation rates. That is because with the highest $ \mu $, the minimum \#$ C_{a} $ and \#$ D $ are lifted the most, then the remaining more cautious cooperators contend with the remaining defectors longer, and hence delays the coming of the growth of defectors. Additionally, it is also known that a higher $ a $ lifts the required fraction of defectors for $ C_{a} $ to start growing. Because of these two reasons, $ \mathscr{C}(C_{a}) $ is prolonged as $ a $ increases. When $ a $ further increases from 0.4 to 0.5, the consequence of reducing the summits of $ C_{a} $ plays the main role again, and $ \mathscr{C}(C_{a}) $ drops accordingly.

Another interesting phenomenon is that $ \mathscr{C}(U_{c}) $ is almost the same as $ \mathscr{C}(C_{a}) $ when $ \mu = 0.05 $. Theoretically, $ \mathscr{C}(C_{a}) $ and $ \mathscr{C}(U_{c}) $ should be similar. However, when $ \mu $ is small, $ U_{c} $ is more likely to raise up more than one time as $ C_{a} $ falling down. In Figure~\ref{Fig3b}, the fraction of trusting cooperators reaches its summit by eliminating cautious cooperators during time step 7 to 13, during which $ C_{a} $ decreases monotonically, but $ U_{c} $ grows up twice at time step 9 and 11 respectively. That is because under lower mutation rate, \#$ D $ and \#$ C_{a} $ are less when $ C_{\bar{a}} $ reaches its summit; then the few remaining defectors are more likely to be paired with $ C_{\bar{a}} $ instead of $ C_{a} $. Namely, the event $ (C_{\bar{a}}, D)|(U_{c}) $ has a higher chance to happen, which stimulates the growth of $ U_{c} $. Therefore, the period of $ \mathbf{y} $ is not completely in line with the period of $ \mathbf{x} $. But when $ \mu $ is very high, more remaining $ D $ and $ C_{a} $ induce a higher probability of the event $ (C_{a}, D)|(U_{c}) $, which prevents $ y_{2} $ from growing multiple times when $ x_{2} $ increases. That's why $ \mathscr{C}(U_{c}) $ and $ \mathscr{C}(D) $ in Figure~\ref{AppenC31} are more similar to each other when $ \mu = 0.05, v = 0.01 $.

% The influence of a on Uc
When $ \mu \neq 0.05 $, $ \mathscr{C}(U_{c}) $ first climbs up and then goes down. When $ a $ increases to 0.2, the inhibition effect plays the main role, it delays the growth of $ C_{a} $ and hence lead to a higher summit of $ U_{c} $, and indirectly increases $ Var(U_{c}) $, as shown in Figure~\ref{Fig7b}. For rule enforcers, the higher summit of $ U_{c} $ takes longer time to reach, $ \mathscr{C}(U_{c}) $ is then prolonged. As $ a $ further increases, $ Var(U_{c}) $ does not change too much, the punishment effect on $ U_{c} $ plays the main role: a heavier punishment accelerates the elimination of $ U_{c} $ by $ C_{a} $, and henceforth a higher $ a $ is accompanied by a shorter $ \mathscr{C}(U_{c}) $ when $ a > 0.2 $.

\begin{figure}[H] % *
    \begin{tabular}{cc}
    \begin{tabular}{c}\begin{subfigure}[c]{0.46\textwidth}
    % \hspace*{-5.5mm}
    \includegraphics[width=7.4cm]{AppenC(2).jpg}
    \caption{$ cov(S_{i}, S_{j}) $}
    \label{Fig7a}
    \end{subfigure}\end{tabular}
    &
    \begin{tabular}{c}\begin{subfigure}[b]{0.46\textwidth}
    % \hspace*{-5.5mm}
    \includegraphics[width=7.4cm]{AppenC(3).jpg}
    \caption{$ Var(S_{i}) $}
    \label{Fig7b}
    \end{subfigure}\end{tabular}
    \end{tabular}
    \caption{The covariance of the fraction of strategies, $ cov(S_{i}, S_{j}) $ ($ S_{i} \text{ and } S_{j} \in \{ C_{a}, C_{\bar{a}}, D, U_{c}\} $) in large scale markets where $ N = 1000 $}, with $ \mu = 0.001, v = 0.005 $. The negative correlation between $ C_{a} $ and $ C_{\bar{a}} $ is weakened with the increase of $ a $, due to the inhibition effect of $ a $ on $ C_{a} $, which shortens the horizontal phase shift between $ x_{1} $ and $ x_{2} $. $ cov(C_{a}, U_{c}) $ is also negative, this negative relationship is strengthened as $ a $ increases, because of the punishment effect of $ a $ on $ U_{c} $ accelerating the elimination of $ U_{c} $ by $ C_{a} $. $ cov(C_{\bar{a}}, D) $ has non-monotonic changes. When $ a $ increases from $ 0.1 $ to $ 0.2 $, the reduced summit of $ C_{a} $ leaves $ C_{\bar{a}} $ a higher chance to co-exist with $ D $, $ cov(C_{\bar{a}}, D) $ therefore increases to a positive number. As $ a $ further increases, \#$ C_{a} $ has a longer time staying at a low level; $ C_{a} $ is more likely to be eliminated by defectors, and $ cov(C_{\bar{a}}, D) $ is hence weakened. $ Var(C_{a}) $ drops (resp. $ Var(U_{c}) $ rises) largely as $ a $ increase from 0.1 to 0.2, which indicates a smaller (resp. larger) wave height of the fraction of $ C_{a} $ (resp. $ U_{c} $) and confirms the lower (resp. higher) summit of $ x_{1} $ (resp. $ y_{2} $).
    \label{Fig7}
\end{figure}

\subsubsection{Cycle length within medium scale markets}\label{AppendixC3_100}
\begin{figure}[H] % *
    \begin{center}
        \includegraphics[height=3.1in]{AppenC(4).jpg}
    \end{center}
    \caption{$ Var(C_{a}) $ and $ Var(U_{c}) $ within medium scale markets where $ N = 100 $. When $ N = 100 $, the event $ (C_{a}, D)|(U_{c}) $ happens with a higher frequency compared with when $ N = 1000 $. This fact amplifies the punishment effect of $ a $, hence the average fraction of the strategy $ U_{c} $ is lower, and $ \mathscr{C}(U_{c}) $ decreases larger as $ a $ increases.}
    \label{Fig8b}
\end{figure}

$ \mathscr{C}(C_{a}) $ is similar under different market scales, but $ \mathscr{C}(U_{c}) $ is different. Let $ R\bigl(\mathscr{C}(U_{c})|(\mu)\bigr) $ be the range of the cycle length of strategy $ U_{c} $ under certain $ \mu $. When $ N = 1000 $, $ R\bigl(\mathscr{C}(U_{c})|(\mu=0.01)\bigr)=0.44 $ and $ R\bigl(\mathscr{C}(U_{c})|(\mu=0.05)\bigr)=0.20 $ , whereas when $ N = 100 $, $ R\bigl(\mathscr{C}(U_{c})|(\mu=0.01)\bigr)=1.27 $ and $ R\bigl(\mathscr{C}(U_{c})|(\mu=0.05)\bigr)=1.43 $. Such results indicate that the punishment effect of $ a $ on $ U_{c} $ is amplified in the medium scale market. This result is owing to the higher frequency of the event $ (C_{a}, D)|(U_{c}) $, through which the punishment effect of $ a $ accelerates the elimination of $ U_{c} $, and eventually shortens $ \mathscr{C}(U_{c}) $.

\subsubsection{The variance of strategies in a medium scale market} \label{AppendixC32}
\begin{figure}[H]
    \begin{center}
        \includegraphics[height=3.1in]{AppenC(5).jpg}
    \end{center}
    \caption{The variance of the population sizes of the coexisting strategies in medium scale markets where $ N = 100 $}.
    \label{AppenC3}
\end{figure}

Compared with in a large scale market, $ Var(U_{c}) $ is much lower in a medium scale market. Because the punishment effect of $ a $ on $ U_{c} $ is amplified by the more often event $ (C_{a}, D)|(U_{c}) $, $ y_{2} $ is more likely to drop from the high level. With the lower summit in each period, $ Var(U_{c}) $ and $ E(y_{2}) $ are lower.

But when $ a \geq 0.2 $ the inhibition effect of $ a $ on $ C_{a} $ plays the main role, which offsets the consequence of decreasing $ E(y_{2}) $ brought by the amplified punishment effect. Compare Figure~\ref{AppenC3} to Figure~\ref{Fig7b}, it can be observed that when $ a $ increases from 0.1 to 0.2, the drops of $ Var(C_{a}) $ within $ N = 100 $ and $ N = 1000 $ are similar; but when $ a $ further increases, $ Var(C_{a}) $ decreases much more in a medium scale market than in a large scale market. When $ N = 100 $, $ Var(C_{a}) $ decreases 42.5\% ($ \mu = 0.01 $) or 30\% ($ \mu = 0.05 $), the numbers are 17.9\% and 18.4\% when $ N = 1000 $. The reduced $ Var(C_{a}) $ indicates the stronger inhibition effect on $C_{a} $, which promotes the growth of $ U_{c} $ and increases $ E(y_{2}) $. That's why $ E(y_{2})|(a \geq 0.2) $ does not show differences in the medium and the large scale market.

\subsection{Supplements of y* within a medium scale market}\label{AppendixC4}

\begin{figure}[H]
    \begin{center}
        \includegraphics[height=3.1in]{AppenC(6).jpg}
    \end{center}
    \caption{The minimum fraction of strategies within medium scale markets where $ N = 100 $. A higher $ \mu $ lifts the minimum fraction of defectors, which makes the defectors cannot permanently being eliminated. The existence of defectors then stimulates the growth of corrupt enforcers, and thereby leads $ \mathbf{y} $ to escape from $ (0.5, 0.5) $ and ultimately evolve to $ \mathbf{y}^*=(0,1) $.}
    \label{AppenC4}
\end{figure}
    
\begin{figure}[H]
    \begin{center}
        \includegraphics[height=3.1in]{AppenC(7).jpg}
    \end{center}
    \caption{The average time for reaching $ \mathbf{y}^* $ in medium scale markets where $ N = 100 $. $ \mathbf{y}^* \in \{(0.5, 0.5), (0, 1) \}$. The punishment effect of $ a $ prevents $ y_{2}^* = 1 $, but the inhibition effect of $ a $ contributs to $ y_{2}^* = 1 $. As $ a $ increase, both these two effects become stronger, the time of reaching $ \mathbf{y}^* $ is accordingly prolonged. When $ a=0.4 $, the time to reach the equilibrium is the longest.}
    \label{AppenC5}
\end{figure}

\subsection{The probability of event \texorpdfstring{$ (C_{a}, D)|(U_{c}) $ is higher within smaller markets}{(Ca, D)|(Uc)}}\label{AppendixC5}
\begin{lemma}
    Given $ \mathbf{x} $ and $ \mathbf{y} $, the event $ (C_{a}, D)|(U_{c}) $ is easier to happen when the scale of the market is smaller, which makes $ y_{2} $ easier to drop from the high level.
    \label{L3}
\end{lemma}

\begin{proof}
    Without loss of generality, let the size of the two markets be $ N_{1} $ and $ N_{2} $ ($ N_{1} < N_{2} $), the corresponding number of rule enforcers are $ M_{1} = N_{1}/2 $ and $ M_{2} = N_{2}/2 $. The probability of the event $ (C_{a}, D)|(U_{c}) $ is $ P\bigl((C_{a}, D)|(U_{c})\bigr) $, we then have:
    \begin{equation}
        \begin{split}
            P_{1}\bigl((C_{a}, D)|(U_{c})\bigr) & = P_{1}(C_{a}, D)P_{1}(U_{c}) \\
                               & = \frac{C_{x_{1}N_{1}}^{\phantom{x}1}C_{y_{2}M_{1}}^{\phantom{x}1}}{C_{N_{1}}^{2}}\\
            P_{2}\bigl((C_{a}, D)|(U_{c})\bigr) & = P_{2}(C_{a}, D)P_{2}(U_{c}) \\
                               & = \frac{C_{x_{1}N_{2}}^{\phantom{x}1}C_{y_{2}M_{2}}^{\phantom{x}1}}{C_{N_{2}}^{2}}\\
            \frac{P_{1}\bigl((C_{a}, D)|(U_{c})\bigr)}{P_{2}\bigl((C_{a}, D)|(U_{c})\bigr)} & = \frac{N_{1}}{N_{2}}\frac{N_{2}-1}{N_{1}-1} \\
                                & > 1
        \end{split}
    \end{equation}
    Thus, the probability of event $ (C_{a}, D)|(U_{c}) $ is higher within a smaller scale market, which decreases the payoff of corrupt enforcers and makes their fraction easier to drop from a high level.
\end{proof}

\pagebreak
\subsection{Discussion about convergence}\label{AppendixC6}
% Conclusion: yes
It is worth considering whether the simulation experiment results of finite markets converge to the analytical results of infinite markets as $ N $ increases. The answer is positive, because the expected payoff of strategies in infinite markets converges to the payoff of strategies in infinite markets. Additionally, the bias in the actual payoff of finite markets caused by randomness tends to zero as $ N $ increases. In the following, we provide the proof supporting our conclusion.

\begin{lemma}\label{Converge1}
    The expected payoff of strategies in finite markets converges to the payoff of strategies in infinite markets as $N$ increases.
\end{lemma}

For better elaboration, we take the payoff of $ U_{c} $ as an example to prove Lemma~\ref{Converge1}. The proof process for other strategies is analogous.

\begin{proof}
    When $ N \to \infty $, the payoff of corrupt enforcers, $ \pi(U_{c}) $ , is a function of $ \mathbf{x} $ as presented in equation (\ref{payoff}). However, in finite markets, the expected payoff of $ U_{c} $ is a function of \#$ C_{a} $, \#$ C_{\bar{a}} $, and \#$ D $. To distinguish between the payoff in finite and infinite markets, we denote the expected payoff in finite markets as $ \pi_{U_{C}} $:
    
    \begin{equation}
    \label{payoffFini} 
        \begin{split}
            \pi_{U_{C}} = 
            & P_{(C_{a},D)}(C_{0}+B-a) + P_{(C_{\bar{a}}, D)}(2C_{0}+B) + P_{(D, D)}(2C_{0}+2B) + \\
            & (1-P_{(C_{a},D)}-P_{(C_{\bar{a}}, D)}-P_{(D, D)})2C_{0}
        \end{split},
    \end{equation}
    where $ N = \#C_{a} + \#C_{\bar{a}} + \#D $, $ P_{(C_{a},D)} = \frac{2\#C_{a}\#D}{N(N-1)} $, $ P_{(C_{\bar{a}}, D)} = \frac{2\#C_{\bar{a}}\#D}{N(N-1)} $, and $ P_{(D, D)} = \frac{\#D(\#D-1)}{N(N-1)} $.

    Since as $ N $ grows, $ P_{(C_{a},D)} \to 2x_{1}x_{3} $, $ P_{(C_{\bar{a}}, D)} \to 2x_{2}x_{3} $, and $ P_{(D, D)} \to x_{3}^2 $, $ \pi_{U_{c}} \to \pi(U_{c}) $. Therefore, the expected payoff of $ U_{c} $ in finite markets converges to the payoff of $ U_{c} $ in infinite markets as $ N $ increases.
\end{proof}

In finite markets, it is not possible to completely eliminate the randomness that introduces bias to the actual payoff. For example, the chance event that every defector is paired with a trusting cooperator can happen and result in the actual payoff for $ U_{c} $ being higher than $ \pi_{U_{c}} $. However, we can prove that the probability of such chance events tends to zero as $ N $ increases.

\begin{lemma}
    \label{Converge2}
    The bias in the actual payoff of finite markets caused by randomness tends to zero as $ N $ increases.
\end{lemma}

We take the aforementioned chance event as an example to prove Lemma~\ref{Converge2}. The proof process for other events that can cause bias is analogous.

\begin{proof}
    Let $ P(D \to C_{\bar{a}}) $ be the probability of the chance event where every defector is paired with a trusting cooperator:

    \begin{equation}
        \label{chanceDC_bara}
        \begin{split}
            P(D \to C_{\bar{a}}) = \frac{2\#D \cdot \mathit{A}_{\#C_{\bar{a}}}^{\#D} \cdot \mathit{A}_{\#C_{a}+(\#C_{\bar{a}} - \# D)}^{2} \cdot \bigl(\frac{N}{2}!\bigr)}{N!}
        \end{split}.
    \end{equation}

    To better understanding the calculation of formula (\ref{chanceDC_bara}), let's imagine that all the players are assigned with an identical number from 1 to N. Then there are $ N! $ ways to arrange the $ N $ players. Among the players, $ \#D $ of them are labeled as $ D $, $ \#C_{\bar{a}} $ of them are labeled as $ C_{\bar{a}} $.
    
    To calculate the probability, we first select a group of players with label $ C_{\bar{a}} $ and pair them with the $ \#D $ number of defectors to form sets of $ (C_{\bar{a}}, D) $. Since we consider $ (C_{\bar{a}}, D) $ and $ (D, C_{\bar{a}}) $ as different pairs, there are in total $ 2\#D \cdot \mathit{A}_{\#C_{\bar{a}}}^{\#D} $ arrangement.
    
    Then the number of remaining cautious cooperators is $ \#C_{a} $, and the number of trusting cooperators is $ \#C_{\bar{a}} - \#D $, they can pair freely. Therefore the number of arrangement for these remaining player is $ \mathit{A}_{\#C_{a}+(\#C_{\bar{a}} - \# D)}^{2} $.
    
    In total, there are $ N/2 $ pairs, and these pairs can be in any order, as a result, the total number of arrangements of the $ N $ players satisfying the condition that every defector is paired with a trusting cooperator is $ 2\#D \cdot \mathit{A}_{\#C_{\bar{a}}}^{\#D} \cdot \mathit{A}_{\#C_{a}+(\#C_{\bar{a}} - \# D)}^{2} \cdot \bigl(\frac{N}{2}!\bigr) $. Finally, dividing this value by $ N! $ will give the precise probability of $ P(D \to C_{\bar{a}}) $.

    Then through calculations, we can easily find that given a specific $ \mathbf{x} $, as $ N $ increases, the probability of $ P(D \approx C_{\bar{a}}) $ decreases rapidly and tends to zero. For example, given $ \mathbf{x} = (0.2, 0.6, 0.2) $, $ P(D \approx C_{\bar{a}}) = 0.119 $ when $ N = 10 $, but when $ N = 50 $, the probability decreases to $ 9.675\times 10^{-22} $. This fact implies that in finite markets, although randomness always exists and leads to the actual payoff deviating from the expected payoff, the likelihood of such deviations decreases as $ N $ increases.
\end{proof}

Accordingly, combining Lemma~\ref{Converge1} and Lemma~\ref{Converge2}, we can conclude that the actual payoff converges to $ \pi_{U_{c}} $, and $ \pi_{U_{c}} $ converges to $ \pi(U_c) $ when $ N\to \infty $.

In fact, this convergence is already evident in the difference in the cycle length of strategies observed in large and medium scale markets. Comparing Figure~\ref{Fig8b} with Figure~\ref{AppenC31}, it is clear that $ \mathscr{C}(S_{i})|N1000 < \mathscr{C}(S_{i})|N100 $. This is because $ \mathbf{x} $ and $ \mathbf{y} $ can sustain longer at the potential equilibria in larger scale markets. As expected, when further increases $ N $, $ \mathscr(S_{i}) $ will decrease to a lower level.

\pagebreak
\subsection{When one rule enforcer monitors multiple pair of players}\label{AppendixC7}

% Question, two ways
In our model, we assume that each pair of players is monitored by one rule enforcer, therefore within finite markets $ M = N/2 $. However, this assumption can be arguable, as in practical life, one rule enforcer can monitor multiple pairs of players. A natural question is how do the original conclusions change when $ M < N/2 $. There are two ways to achieve $ M < N/2 $: by reducing the number of rule enforcers or by expanding the market scale to include more players; for the convenience of elaboration, we do not consider changing these two factors at the same time.

% For infinite markets
In the case of infinite markets, the replicator dynamics solely depend on $ \mathbf{x} $ and $ \mathbf{y} $, which are unaffected by either of these two ways. Therefore, when $ M < N/2 $, the player-enforcer dynamics remain unchanged, and the original conclusions still hold. For finite market, however, the evolution of strategies might be affected because of the affected mutation rate and the changed probability of events. Next, we discuss the results arise from reducing $ M $ and increasing $ N $ in finite markets respectively.

% Reducing $ M $ in finite markets
By reducing $ M $ while keeping $ N $ constant, the number of pairs of players that one rule enforcer monitors increases. This change itself has no influence the evolution of strategy profiles in a finite population, provided that each rule enforcer is randomly assigned to the same number of pairs. This assumption ensures that the assignment of each rule enforcer to $ N/(2M') $ pairs of players produces the same results as the alternative approach: duplicating the $ M' $ rule enforcers $ (N/2)/M' $ times, thereby expanding the number of rule enforcers to $ N/2 $, and assigning each of them one pair.

However, the evolution process is influenced if the reduction of $ M $ affects the effectiveness of the mutation rate $ v $ and impedes the invasions of randomly exploring rule enforcers. To illustrate this, we consider two representative examples: $ M = 50 $ and $ M = 5 $, while keeping $ N = 1000 $. In the former case, under high mutation rates, only the number of monitored pairs changes, while in the latter case, the effect of the mutation rate changes.

% Not afffect v
Consider a large scale market where $ N = 1000 $ with 50 rule enforcers. Under high mutation rates, randomly matching 500 rule enforcers with the 500 pairs of players is the same as assigning 10 pairs of players to each of the 50 rule enforcers. In both scenarios, $ \mathbf{x} $ and $ \mathbf{y} $ exhibit cyclic dominance due to the invasions caused by the mutation rate. The mean value of the fraction of specific strategies under $ M=50 $ are presented in Figure~\ref{AppenC8-M50}, which aligns with the original results under $ M=500 $ (Figure~\ref{Fig5}).

\begin{figure}[H]
    \begin{center}
        \includegraphics[height=3.1in]{AppenC(8)-M50.jpg}
    \end{center}
    \caption{The mean value of the fraction of specific strategies ($ E(\# S_{i}/N) $) when $ N = 1000 $, $ M = 50 $. Since reducing $ M $ from 500 to 50 does not change the effects of $ v $, random exploration by rule enforcers can always lead to invasions, resulting in cyclic dominance of $ \mathbf{y} $. In each round of evolution, assigning each rule enforcers to 10 pairs of players is equivalent to expanding the 50 rule enforcers to 500 and randomly matching them with the 500 pairs of players. Consequently, the evolution results are the same as those obtained when $ M = 500 $.}\label{AppenC8-M50}
\end{figure}

% Afffect v
However, under low mutation rates, $ M = 50 $ leads to $ \mathbf{y}^* = (0, 1) $, which is distinct from the cyclic dominance of $ \mathbf{y} $ observed in $ M=500 $. The reason is that $ M = 50 $ makes the low mutation rates ($ v = 0.001 $ and $ v=0.005 $) lose their effectiveness, as no randomly exploring rule enforcers can invade once $ \mathbf{y}^* = (0,1) $ is reached. As expected, further reducing $ M $ to 5 makes $ \mathbf{y}^* $ always reachable under both high and low mutation rates. The relative frequency of $ \mathbf{y}^* $ with $ M = 5 $ among the 500 repeated experiments is presented in Figure~\ref{AppenC8-M5}. In summary, reducing $ M $ alone does not influence the evolution in finite markets unless it affects the mutation rate, as too small $ M $ impedes the presence of randomly exploring rule enforcers.

\begin{figure}[H]
    \begin{center}
        \includegraphics[height=3.1in]{AppenC(8)-M5.jpg}
    \end{center}
    \caption{The relative frequency of specific equilibrium of rule enforcers among the 500 repetitions, with $ M = 5 $ and $ N = 1000 $. In contrast to the cyclic dominance observed in Figure~\ref{Fig6} where $ M = 500 $, reducing $ M $ to $ 5 $ actually makes the mutation rate of rule enforcers, $ v $, lose effectiveness. As a result, the equilibrium $ \mathbf{y}^* $ is always reachable. When comparing these results to those obtained in small scale markets where $ M $ remains the same but $ N = 10 $ (Figure~\ref{Fig4}), $ \mathbf{y}^* = (0.5, 0.5) $ does not occur when $ N = 1000 $. This difference results from the persistent presence of defectors, which stimulates the growth of $ U_{c} $ and drives $ \mathbf{y} = (0.5, 0.5) $ to $ \mathbf{y}^*= (0, 1) $.}
    \label{AppenC8-M5}
\end{figure}

% Increasing N in finite markets
By the other way that increasing $ N $ with controlling $ M $, the number of randomly exploring players might increase. For example, Figure~\ref{AppenC8-M5} and Figure~\ref{Fig4} represent the results under $ N = 10 $ and $ N = 1000 $ respectively, while keeping $ M = 5 $. It can be observed that $ \mathbf{y}^* = (0.5, 0.5) $ does not occur when $ N = 1000 $, in contrast to the results when $ N=10 $. This is because the larger $ N $ prevents the complete elimination of defectors, which stimulates the growth of corrupt enforcers and leads $ \mathbf{y} = (0.5, 0.5) $ to evolve into $ \mathbf{y}^* = (0, 1) $. Therefore, the only opportunity for the market to evolve into a desirable state is that each corrupt rule enforcers being matched with the pair $ (C_{a}, D) $ during the early stages of the evolution. This matching drives the rapid evolution of $ \mathbf{y} $ towards $ \mathbf{y}^* = (1, 0) $.

In this circumstance, a high $ v $ in turn prevents the elimination of corrupt enforcers at the early stages, as it increases the probability of honest enforcers exploring strategy $ U_{c} $. That's why the probability of evolving into a good market that contains only honest enforcers are even higher when $ v < \mu $ in Figure~\ref{AppenC8-M5}. This result contradicts to our previous conclusion that higher $ v $ is always beneficial, since the expanded $ N $ essentially prevents the potential equilibrium $ \mathbf{y}^* = (0.5, 0.5) $ under $ v > \mu $.

The expansion of $ N $ also brings another change: the lower probability of event $ (C_{a}, D)|(U_{c}) $. This results in different mean value of the fraction of strategies between the case of $ N = 100 $ (Figure~\ref{Fig6}) and $ N = 1000 $ (Figure~\ref{AppenC8-M50}). Since the mechanism behind this difference has been illustrated in Section~\ref{Dynamics of N100}, we will not delve into it further here.

% summary
To summarize, when one rule enforcer monitors multiple pairs of players in infinite markets, all the original conclusions still hold. In finite markets, reducing $ M $ and increasing $ N $ have different influence. By reducing $ M $, as long as it does not affect the mutation rate of rule enforcers, all the original conclusions hold, otherwise $ \mathbf{y}^* $ becomes always reachable. Increasing $ N $ can impede $ \mathbf{y}^* = (0.5, 0.5) $ and eventually lead to $ \mathbf{y}^* = (0, 1) $ when $ \mathbf{y}^* $ is reachable; when $ \mathbf{y}^* $ is unreachable, increasing $ N $ influence $ E(\# S_{i}/N) $, but does not alter the original conclusions regarding the influence of $ a $ and $ v $.

\bibliographystyle{elsarticle-num}
% % \section{\refname}
% \bibliography{cas-refs}
\begin{thebibliography}{00}
\bibitem{sandholm2010local}Sandholm, W. Local stability under evolutionary game dynamics. {\em Theoretical Economics}. \textbf{5}, 27-50 (2010)
\bibitem{hines1980three}Hines, W. Three characterizations of population strategy stability. {\em Journal Of Applied Probability}. \textbf{17}, 333-340 (1980)
\end{thebibliography}


\end{document}